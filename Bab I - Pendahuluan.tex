% ==========================================
% BAB I PENDAHULUAN
% ==========================================
\chapter{PENDAHULUAN}
\label{chap:pendahuluan}
% --- Latar Belakang ---
\section{Latar Belakang}
Peran kantin tidak hanya menyediakan makanan, tetapi juga mendukung keamanan dan kenyamanan konsumsi dari para konsumennya. Kantin merupakan fasilitas penting dalam mendukung aktivitas civitas akademika di berbagai institusi. Selain menyediakan kebutuhan konsumsi harian, kantin memiliki peran dalam menjaga keamanan pangan, efisiensi operasional, dan keberlanjutan lingkungan.

Pada konteks kantin modern, \textit{food waste} tidak hanya berdampak pada lingkungan, namun juga memiliki implikasi terhadap tiga aspek penting. Data FAO dan UNEP menunjukkan masalah \textit{food waste} telah menjadi isu global yang signifikan, dengan 1,3 miliar ton makanan terbuang setiap tahun secara global, dan 25\% diantaranya merupakan \textit{food waste} yang dapat dicegah dari sektor layanan makanan. Dalam menjawab tantangan ini, terdapat tiga strategi utama yang perlu diterapkan, yakni meningkatkan biaya operasional, mengurangi nilai gizi dan keamanan pangan, serta menurunkan efisiensi proses bisnis.

Di tingkat \textit{global}, Codex Alimentarius (2020) dan pendekatan HACCP (\textit{Hazard Analysis and Critical Control Points}) menetapkan standar untuk memastikan keamanan pangan. Keputusan Menteri Kesehatan No. 942/2003 dan No. 1429/2006 menyatakan bahwa kantin ideal seharusnya mampu menjamin kelayakan makanan, higienitas penyajian, serta pengelolaan limbah pangan dengan baik selaras dengan prinsip keamanan pangan modern. Strategi identifikasi bahaya mencakup pemantauan kondisi makanan sepanjang siklus penyajiannya, khususnya terkait deteksi makanan yang masih baik (\textit{not spoiled}) serta makanan yang sudah tidak layak konsumsi (\textit{spoiled}).

Smart Canteen yang ideal membutuhkan sistem pemantauan kondisi makanan yang objektif dan sesuai standar nasional serta global untuk memastikan keamanan pangan dan mengidentifikasi terjadinya pembusukannya secara \textit{real-time} dan otomatis.

% --- Rumusan Masalah ---
\section{Rumusan Masalah}
Rumusan Masalah berisi masalah utama yang dibahas dalam tugas akhir. Rumusan masalah yang baik memiliki struktur sebagai berikut:
\begin{enumerate}
\item	Pokok persoalan dari kondisi atau situasi yang ada saat ini. Dengan kata lain, jelaskan kelemahan atau kekurangan dari kondisi, situasi, atau solusi yang dijelaskan pada latar belakang. Ini merupakan pokok rumusan masalah.
\item	Elaborasi lebih lanjut urgensi penyelesaian masalah tersebut (mengapa penting untuk diselesaikan dan akibat yang dapat terjadi jika tidak diselesaikan).
\item	Usulan singkat terkait dengan solusi yang ditawarkan untuk menyelesaikan persoalan.
Penting untuk diperhatikan bahwa persoalan yang dideskripsikan pada subbab ini akan dipertanggungjawabkan di bab Evaluasi (apakah terselesaikan atau tidak).
\end{enumerate}

% --- Tujuan ---
\section{Tujuan}
Tuliskan tujuan utama dan/atau tujuan detail yang akan dicapai dalam pelaksanaan tugas akhir. Fokuskan pada hasil akhir yang ingin diperoleh setelah tugas akhir diselesaikan, terkait dengan penyelesaian persoalan pada rumusan masalah. Penting untuk diperhatikan bahwa tujuan yang dideskripsikan pada subbab ini akan dipertanggungjawabkan di akhir pelaksanaan tugas akhir apakah tercapai atau tidak. Tuliskan kriteria keberhasilan tugas akhir ini.

% --- Batasan Masalah ---
\section{Batasan Masalah}
Tuliskan batasan-batasan yang diambil dalam pelaksanaan tugas akhir. Batasan ini dapat dihindari (bersifat opsional, tidak perlu ada) jika topik atau judul tugas akhir dibuat cukup spesifik.

% --- Metodologi Pengerjaan TA ---
\section{Metodologi}
Tuliskan semua tahapan yang akan dilalui selama pelaksanaan tugas akhir. Tahapan ini spesifik untuk menyelesaikan persoalan tugas akhir. Khusus untuk penyusunan proposal ini, jelaskan secara detail:
\begin{enumerate}
\item	Tahapan investigasi pengumpulan fakta di latar belakang untuk merumuskan masalah.
\item	Langkah-langkah pencarian, pengelompokan, dan penapisan literatur atau sumber informasi untuk mengumpulkan informasi yang relevan tentang topik yang diangkat, termasuk teori (konsep atau teori apa saja yang perlu dicari), hal-hal yang telah dicapai oleh orang lain (cara mencari dan kata kuncinya), dan berbagai informasi pendukung, untuk mencari solusi terhadap masalah yang dibahas. Gunakan metodologi yang tepat dalam menggali informasi dan dokumentasikan prosesnya (termasuk rekaman wawancara atau survei) di dalam Lampiran, termasuk tautan ke video atau foto. Hasil penggalian informasi ini akan dijelaskan secara sistematis di Bab II Studi Literatur.
\end{enumerate}