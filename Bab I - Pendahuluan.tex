% ==========================================
% BAB I PENDAHULUAN
% ==========================================
\chapter{PENDAHULUAN}
\label{chap:pendahuluan}
% --- Latar Belakang ---
\section{Latar Belakang}

Dalam operasional kantin, salah satu tantangan terbesar adalah memastikan bahwa makanan yang disajikan tetap aman dan layak konsumsi sepanjang siklus penyajiannya. Makanan yang telah melalui proses masak, penyimpanan, dan penyajian memiliki potensi mengalami penurunan kualitas hingga pembusukan (\textit{spoilage}), yang sering kali tidak terlihat secara kasat mata. Kondisi ini berbahaya karena makanan yang terlihat baik secara visual bisa saja telah terkontaminasi atau mengalami perubahan kimia dan mikrobiologis yang membahayakan kesehatan konsumen.

Di banyak kantin, termasuk dalam ekosistem Smart Canteen yang sedang berkembang, penilaian terhadap kesegaran makanan masih dilakukan secara manual, berbasis indra penciuman, pengamatan visual, atau pengalaman subjektif penjual. Metode ini tidak hanya kurang akurat dan tidak konsisten, tetapi juga tidak memenuhi standar keamanan pangan yang menuntut proses pemantauan yang objektif, terukur, dan terdokumentasi. Ketidakmampuan mendeteksi \textit{spoilage} secara dini dapat meningkatkan risiko keracunan makanan, menciptakan ketidaknyamanan konsumen, serta berdampak negatif pada reputasi layanan kantin.

Isu ini semakin relevan mengingat data FAO dan UNEP menunjukkan bahwa pembusukan pangan tidak hanya menimbulkan risiko kesehatan, tetapi juga menyumbang porsi signifikan dari \textit{food waste} global. Lebih dari 25\% \textit{food waste} pada sektor layanan makanan sebenarnya dapat dicegah melalui sistem pemantauan kondisi makanan yang lebih akurat dan ilmiah.

Di tingkat internasional, Codex Alimentarius dan prinsip \textit{Hazard Analysis and Critical Control Points} (HACCP) menekankan pentingnya pengendalian titik-titik kritis, termasuk pemantauan kualitas makanan setelah dimasak dan disajikan. Regulasi nasional seperti Keputusan Menteri Kesehatan No. 942/2003 dan No. 1429/2006 juga mengharuskan setiap kantin memastikan kelayakan dan keamanan makanan yang disajikan. Namun dalam praktiknya, teknologi yang mendukung pemantauan \textit{real-time} terhadap kesegaran makanan belum banyak diimplementasikan, khususnya pada fase \textit{post-dining} di mana risiko \textit{spoilage} meningkat.

Perkembangan teknologi sensor gas dan \textit{machine learning} membuka peluang baru untuk mendeteksi gas volatil seperti ammonia, hydrogen sulfide, methane, dan VOC lain yang menjadi indikator kuat pembusukan makanan. Pendekatan ini menawarkan cara \textit{non-destructive}, objektif, dan \textit{real-time} untuk menilai kondisi makanan. Dengan mengintegrasikan sistem pendeteksi berbasis sensor MQ dalam ekosistem Smart Canteen, penilaian kesegaran makanan dapat dilakukan secara otomatis untuk mendukung keputusan seperti menyimpan, mendonasikan, atau membuang makanan, sehingga keamanan pangan terjamin dan \textit{food waste} dapat ditekan.

% --- Rumusan Masalah ---
\section{Rumusan Masalah}

Penelitian ini berfokus untuk mengembangkan model \textit{machine learning} berbasis sensor gas MQ yang mampu mengklasifikasikan makanan sebagai \textit{fresh} atau \textit{spoiled} secara objektif.

Rumusan masalah penelitian adalah sebagai berikut:

\begin{enumerate}
\item	Bagaimana mendeteksi tingkat kesegaran makanan secara objektif tanpa inspeksi manual pada lingkungan Smart Canteen?

\item	Bagaimana membangun dan mengevaluasi model \textit{machine learning} yang mampu mengklasifikasikan makanan ke dalam alur \textit{fresh} atau \textit{spoiled} berdasarkan data gas dari sensor MQ?

\item	Bagaimana mengintegrasikan hasil klasifikasi kesegaran makanan ke dalam alur Food Waste Management untuk mendukung keputusan \textit{storing}, \textit{donation}, dan \textit{disposal}?
\end{enumerate}

% --- Tujuan ---
\section{Tujuan}

Tujuan penelitian ini adalah:

\begin{enumerate}
\item	Mengembangkan sistem pemantauan kesegaran makanan berbasis \textit{multi-gas sensor} MQ yang mampu mengukur kondisi makanan secara objektif.

\item	Membangun dan mengevaluasi model \textit{machine learning} yang dapat mengklasifikasikan makanan \textit{fresh} atau \textit{spoiled} berdasarkan data sensor gas dari sensor MQ.

\item	Merancang integrasi hasil klasifikasi ke dalam \textit{workflow} Smart Canteen untuk mendukung keputusan pada proses \textit{storing}, \textit{donation}, dan \textit{disposal}.
\end{enumerate}

% --- Batasan Masalah ---
\section{Batasan Masalah}

Penelitian ini memiliki batasan-batasan sebagai berikut:

\begin{enumerate}
\item	Penelitian ini hanya berfokus pada klasifikasi dua kondisi makanan, yaitu \textit{fresh} dan \textit{spoiled}, tanpa mencakup kategori \textit{intermediate} seperti \textit{almost spoiled} atau penilaian tingkat kerusakan yang lebih terperinci.

\item	Data yang digunakan berasal dari pembacaan \textit{multi-gas sensor} MQ (MQ8, MQ135, MQ4, MQ9, MQ2, MQ3) dalam bentuk nilai analog dan digital, tanpa melibatkan data visual, aroma manusia, suhu, kelembapan, atau sensor IoT tambahan lainnya.

\item	Ruang lingkup sampel pangan dibatasi pada empat komoditas, yaitu daging ayam mentah, daging ikan mentah, buah pisang, dan buah apel. Pemilihan ini dilakukan karena komoditas tersebut umum dijual di kantin dan menghasilkan pola VOC yang dapat dideteksi oleh sensor MQ.

\item	Model \textit{machine learning} yang dibangun hanya bertujuan mendeteksi kondisi makanan berdasarkan pola gas, tanpa mencakup penjelasan penyebab kerusakan, prediksi waktu kedaluwarsa, atau estimasi \textit{shelf-life}.

\item	Integrasi sistem dengan \textit{workflow} Smart Canteen dibatasi pada rekomendasi keputusan (\textit{storing}, \textit{donation}, \textit{disposal}), tanpa mencakup implementasi penuh seperti sistem otomasi perangkat keras kantin.

\item	Pengujian sistem dilakukan menggunakan \textit{dataset} hasil eksperimen sendiri, yaitu data gas VOC yang dikumpulkan secara langsung selama proses pembusukan empat komoditas tersebut. Penelitian ini tidak menggunakan \textit{dataset} resmi, termasuk \textit{dataset} dari FAO atau sumber publik lainnya, karena belum tersedia \textit{dataset} standar untuk pola emisi VOC pada makanan.
\end{enumerate}

% --- Metodologi Pengerjaan TA ---
\section{Metodologi}

Metode yang akan dipilih dalam penelitian ini adalah CRISP-DM (\textit{Cross Industry Standard Process for Data Mining}) sebagai berikut:

\subsection{Fase 1: Understanding (\textit{Pemahaman})}

\subsubsection{\textit{Business Understanding}}
Identifikasi permasalahan \textit{food spoilage} dalam Smart Canteen dan kebutuhan sistem pemantauan kesegaran makanan yang objektif berbasis sensor gas.

\subsubsection{\textit{Data Understanding}}
Eksplorasi \textit{dataset} sensor MQ yang mencakup pembacaan sinyal gas pada makanan dalam kondisi \textit{fresh} dan \textit{spoiled}, termasuk analisis karakteristik data dan distribusinya.

\subsection{Fase 2: Preparation \& Modelling (\textit{Persiapan dan Pemodelan})}

\subsubsection{\textit{Data Preparation}}
Melakukan \textit{filtering}, \textit{transform}, penyesuaian data, \textit{splitting} data pelatihan dan pengujian, serta pemilihan fitur yang relevan dari pembacaan sensor MQ.

\subsubsection{\textit{Data Modelling}}
Membangun model klasifikasi menggunakan Logistic Regression, Random Forest, Decision Tree, dan KNN untuk mengklasifikasikan makanan sebagai \textit{fresh} atau \textit{spoiled} berdasarkan data sensor.

\subsection{Fase 3: Evaluating (\textit{Evaluasi})}

\subsubsection{\textit{Data Evaluation}}
Mengevaluasi performa model menggunakan metrik \textit{Accuracy}, \textit{Precision}, \textit{Recall}, \textit{F1-Score}, \textit{Confusion Matrix}, dan \textit{Cross-Validation}.

\vspace{0.5cm}

CRISP-DM cocok digunakan dalam penelitian ini karena metodologinya memberikan alur kerja yang terstruktur mulai dari pemahaman kebutuhan bisnis hingga evaluasi model.