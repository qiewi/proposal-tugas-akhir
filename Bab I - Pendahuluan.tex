% ==========================================
% BAB I PENDAHULUAN
% ==========================================
\chapter{PENDAHULUAN}
\label{chap:pendahuluan}
% --- Latar Belakang ---
\section{Latar Belakang}
Peran kantin tidak hanya menyediakan makanan, tetapi juga mendukung keamanan dan kenyamanan konsumsi dari para konsumennya. Kantin merupakan fasilitas penting dalam mendukung aktivitas civitas akademika di berbagai institusi. Selain menyediakan kebutuhan konsumsi harian, kantin memiliki peran dalam menjaga keamanan pangan, efisiensi operasional, dan keberlanjutan lingkungan.

Pada konteks kantin modern, \textit{food waste} tidak hanya berdampak pada lingkungan, namun juga memiliki implikasi terhadap tiga aspek penting. Data FAO dan UNEP menunjukkan masalah \textit{food waste} telah menjadi isu global yang signifikan, dengan 1,3 miliar ton makanan terbuang setiap tahun secara global, dan 25\% diantaranya merupakan \textit{food waste} yang dapat dicegah dari sektor layanan makanan. Dalam menjawab tantangan ini, terdapat tiga strategi utama yang perlu diterapkan, yakni meningkatkan biaya operasional, mengurangi nilai gizi dan keamanan pangan, serta menurunkan efisiensi proses bisnis.

Di tingkat \textit{global}, Codex Alimentarius (2020) dan pendekatan HACCP (\textit{Hazard Analysis and Critical Control Points}) menetapkan standar untuk memastikan keamanan pangan. Keputusan Menteri Kesehatan No. 942/2003 dan No. 1429/2006 menyatakan bahwa kantin ideal seharusnya mampu menjamin kelayakan makanan, higienitas penyajian, serta pengelolaan limbah pangan dengan baik selaras dengan prinsip keamanan pangan modern. Strategi identifikasi bahaya mencakup pemantauan kondisi makanan sepanjang siklus penyajiannya, khususnya terkait deteksi makanan yang masih baik (\textit{not spoiled}) serta makanan yang sudah tidak layak konsumsi (\textit{spoiled}).

Smart Canteen yang ideal membutuhkan sistem pemantauan kondisi makanan yang objektif dan sesuai standar nasional serta global untuk memastikan keamanan pangan dan mengidentifikasi terjadinya pembusukannya secara \textit{real-time} dan otomatis.

% --- Rumusan Masalah ---
\section{Rumusan Masalah}

Penelitian ini berfokus untuk mengembangkan model \textit{machine learning} berbasis sensor gas MQ yang mampu mengklasifikasikan makanan sebagai \textit{fresh} atau \textit{spoiled} secara objektif.

Rumusan masalah penelitian adalah sebagai berikut:

\begin{enumerate}
\item	Bagaimana mendeteksi tingkat kesegaran makanan secara objektif tanpa inspeksi manual pada lingkungan Smart Canteen?

\item	Bagaimana membangun dan mengevaluasi model \textit{machine learning} yang mampu mengklasifikasikan makanan ke dalam alur \textit{fresh} atau \textit{spoiled} berdasarkan data gas dari sensor MQ?

\item	Bagaimana mengintegrasikan hasil klasifikasi kesegaran makanan ke dalam alur Food Waste Management untuk mendukung keputusan \textit{storing}, \textit{donation}, dan \textit{disposal}?
\end{enumerate}

% --- Tujuan ---
\section{Tujuan}

Tujuan penelitian ini adalah:

\begin{enumerate}
\item	Mengembangkan sistem pemantauan kesegaran makanan berbasis \textit{multi-gas sensor} MQ yang mampu mengukur kondisi makanan secara objektif.

\item	Membangun dan mengevaluasi model \textit{machine learning} yang dapat mengklasifikasikan makanan \textit{fresh} atau \textit{spoiled} berdasarkan data sensor gas dari sensor MQ.

\item	Merancang integrasi hasil klasifikasi ke dalam \textit{workflow} Smart Canteen untuk mendukung keputusan pada proses \textit{storing}, \textit{donation}, dan \textit{disposal}.
\end{enumerate}

% --- Batasan Masalah ---
\section{Batasan Masalah}

Penelitian ini memiliki batasan-batasan sebagai berikut:

\begin{enumerate}
\item	Penelitian ini hanya berfokus pada klasifikasi dua kondisi makanan, yaitu \textit{fresh} dan \textit{spoiled}, tanpa mencakup kategori \textit{intermediate} seperti \textit{almost spoiled} atau penilaian tingkat kerusakan.

\item	Data yang digunakan berasal dari pembacaan \textit{multi-gas sensor} MQ (MQ8, MQ135, MQ4, MQ9, MQ2, MQ3) dalam bentuk nilai \textit{analog} digital, tanpa melibatkan data visual, aroma manusia, suhu, kelembapan, atau sensor IoT tambahan lainnya.

\item	Model \textit{machine learning} yang dibangun hanya bertujuan mendeteksi kondisi makanan berdasarkan pola gas, tanpa mencakup penjelasan penyebab kerusakan, prediksi waktu kadaluarsa, atau estimasi \textit{shelf-life}.

\item	Integrasi sistem dengan \textit{workflow} Smart Canteen dibatasi pada rekomendasi keputusan (\textit{storing}, \textit{donation}, \textit{disposal}), tanpa mencakup implementasi penuh pada sistem operasional atau perangkat keras kantin.

\item	Pengujian sistem dilakukan menggunakan \textit{dataset} hasil eksperimen sensor yang tersedia, tanpa melakukan pengujian skala besar pada kondisi kantin nyata ataupun \textit{dataset} resmi dari FAO.
\end{enumerate}

% --- Metodologi Pengerjaan TA ---
\section{Metodologi}

Metode yang akan dipilih dalam penelitian ini adalah CRISP-DM (\textit{Cross Industry Standard Process for Data Mining}) sebagai berikut:

\subsection{Fase 1: Understanding (\textit{Pemahaman})}

\subsubsection{\textit{Business Understanding}}
Identifikasi permasalahan \textit{food spoilage} dalam Smart Canteen dan kebutuhan sistem pemantauan kesegaran makanan yang objektif berbasis sensor gas.

\subsubsection{\textit{Data Understanding}}
Eksplorasi \textit{dataset} sensor MQ yang mencakup pembacaan sinyal gas pada makanan dalam kondisi \textit{fresh} dan \textit{spoiled}, termasuk analisis karakteristik data dan distribusinya.

\subsection{Fase 2: Preparation \& Modelling (\textit{Persiapan dan Pemodelan})}

\subsubsection{\textit{Data Preparation}}
Melakukan \textit{filtering}, \textit{transform}, penyesuaian data, \textit{splitting} data pelatihan dan pengujian, serta pemilihan fitur yang relevan dari pembacaan sensor MQ.

\subsubsection{\textit{Data Modelling}}
Membangun model klasifikasi menggunakan Logistic Regression, Random Forest, Decision Tree, dan KNN untuk mengklasifikasikan makanan sebagai \textit{fresh} atau \textit{spoiled} berdasarkan data sensor.

\subsection{Fase 3: Evaluating (\textit{Evaluasi})}

\subsubsection{\textit{Data Evaluation}}
Mengevaluasi performa model menggunakan metrik \textit{Accuracy}, \textit{Precision}, \textit{Recall}, \textit{F1-Score}, \textit{Confusion Matrix}, dan \textit{Cross-Validation}.

\vspace{0.5cm}

CRISP-DM cocok digunakan dalam penelitian ini karena metodologinya memberikan alur kerja yang terstruktur mulai dari pemahaman kebutuhan bisnis hingga evaluasi model.