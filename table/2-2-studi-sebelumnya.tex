\begin{longtable}{|p{2.5cm}|p{2.8cm}|p{3.2cm}|p{3.2cm}|}
\caption{Studi Sebelumnya}\label{tbl:studi-sebelumnya} \\
\hline
\textbf{Judul (Penulis)} & \textbf{Metodologi} & \textbf{Kontribusi Utama} & \textbf{Keterbatasan} \\
\hline
\endfirsthead

\caption{Studi Sebelumnya (lanjutan)} \\
\hline
\textbf{Judul (Penulis)} & \textbf{Metodologi} & \textbf{Kontribusi Utama} & \textbf{Keterbatasan} \\
\hline
\endhead

\hline
\multicolumn{4}{r}{\textit{Bersambung ke halaman berikutnya}} \\
\endfoot

\hline
\endlastfoot

Food Waste Management Strategy with Green and Digital Technology (Vitri Tundjungsih, 2025) &
Perbandingan 5 pendekatan digital: Computer Vision, Sensor-based Monitoring, Sensor Fusion + ML, Object Detection, dan \textit{Rule-based Systems}; menggunakan kerangka AI + IoT untuk \textit{edibility} detection dan \textit{smart sorting} &
Memberikan model strategi yang menggabungkan AI, IoT, dan \textit{green technology} untuk mendeteksi kelayakan makanan (appearance, smell, gas, temperature) dan menentukan jalur pemanfaatan (donation, animal feed, composting) &
Pendekatan sebagian besar konseptual; model Computer Vision \& sensor fusion belum diimplementasikan secara penuh. Membutuhkan dataset besar dan berkualitas tinggi. \\
\hline
Meat and Fish Freshness Inspection System Based on Odor Sensing (Najam ul Hasan, Naveed Ejaz, Waleed Ejaz, Hyung Seok Kim — 2012) &
Pengembangan \textit{electronic nose} berbasis MOS gas sensor array (8 sensor); preprocessing (moving average filtering); \textit{feature extraction} (steady-state math); perbandingan 3 model klasifikasi (ANN, SVM, KNN) &
Menghasilkan sistem inspeksi kesegaran daging rendah biaya yang mampu mendeteksi daging busuk melalui pola gas dengan akurasi tertinggi (96,6\%); menyediakan arsitektur lengkap (handheld device + server) untuk inspeksi makanan di butcher shop &
Hanya diuji pada dua jenis makanan (beef dan fish); tidak menangani klasifikasi multi-item; membutuhkan chamber dan kondisi eksperimen terkontrol; model ANN/SVM kurang optimal; belum diuji pada lingkungan kantin atau makanan olahan \\
\hline
\textit{Federated learning}-driven IoT system for automated freshness monitoring in resource-constrained vending carts (Stephan \textit{dkk.}, 2025) &
IoT + gas sensors MQ2 (Hydrogen), MQ9 (CO), MQ135 (Ammonia), MQ138 (Toluene) untuk memantau emisi \textit{spoilage}. Decision Tree classifier digunakan sebagai local model setiap cart &
Mengonfirmasi efektivitas MO-series gas sensors untuk mendeteksi sinyal kimia pembusukan secara \textit{real-time}. Menunjukkan bahwa kombinasi gas sensor + ML dapat menghasilkan klasifikasi freshness akurasi sangat tinggi (hingga 99.97\%). &
Sistem sangat kompleks (FL + ensemble), kurang cocok untuk implementasi UMKM/kantin sederhana. \\
\hline
\end{longtable}
