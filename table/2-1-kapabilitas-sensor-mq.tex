\begin{minipage}{\textwidth}
\begin{table}[H]
  \centering
  \caption{Ringkasan Kapabilitas Sensor MQ Berdasarkan Literatur 2015–2024}
  \label{tbl:sensor-mq-capabilities}
  \begin{tabular}{|p{1cm}|p{2.2cm}|p{2.2cm}|p{3.2cm}|p{2.8cm}|}
    \hline
    \textbf{Sensor} & \textbf{Target Gas} & \textbf{Rentang Deteksi} & \textbf{Aplikasi pada Spoilage} & \textbf{Catatan Keterbatasan} \\
    \hline
    MQ2 & LPG, H$_2$, metana, CO & 200–10,000 ppm & Mendeteksi methane dan hydrogen di tahap awal pembusukan & Cross-sensitivity, pengaruh kelembapan \\
    \hline
    MQ3 & Etanol, metanol & 10-500 ppm & Sensitif terhadap alkohol dari fermentasi dan pembusukan mikroba & Selektivitas terbatas pada alkohol \\
    \hline
    MQ4 & Metana, gas natural & 200–10,000 ppm & Mendeteksi methane dari spoilage anaerob & Sensitivitas terbatas pada methane \\
    \hline
    MQ8 & Hidrogen & 100–10,000 ppm & Mendeteksi hydrogen sebagai gas metabolik dari aktivitas mikroba & Respon rendah terhadap gas spoilage lain \\
    \hline
    MQ9 & Carbon monoksida, metana, LPG & 10–10,000 ppm (CO) & Mendeteksi CO dan methane dalam proses degradasi bahan organik & Cross-sensitivity terhadap berbagai gas \\
    \hline
    MQ135 & NH3 (H$_2$S) & 10-1,000 ppm & Sangat sensitif terhadap Ammonia pada daging dan seafood & Selektivitas rendah, mudah drift \\
    \hline
  \end{tabular}
\end{table}
\end{minipage}
