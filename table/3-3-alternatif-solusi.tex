\begin{longtable}{|p{1.5cm}|p{3.5cm}|p{3.5cm}|p{3.5cm}|}
\caption{Alternatif Solusi untuk Sistem Pemantauan Kesegaran Makanan}\label{tbl:alternatif-solusi} \\
\hline
\textbf{No.} & \textbf{Solusi} & \textbf{Kelebihan} & \textbf{Kekurangan} \\
\hline
\endfirsthead

\caption{Alternatif Solusi untuk Sistem Pemantauan Kesegaran Makanan (lanjutan)} \\
\hline
\textbf{No.} & \textbf{Solusi} & \textbf{Kelebihan} & \textbf{Kekurangan} \\
\hline
\endhead

\hline
\multicolumn{4}{r}{\textit{Bersambung ke halaman berikutnya}} \\
\endfoot

\hline
\endlastfoot

1 & \textbf{\textit{Rule-Based System}} (\textit{threshold} manual per sensor MQ) & Mudah diimplementasikan pada skala kecil, tidak butuh \textit{training} model, eksekusi sangat cepat. & Tidak akurat untuk pola gas yang kompleks, tidak \textit{scalable}, \textit{threshold} tidak stabil antar makanan. \\
\hline
2 & \textbf{\textit{Single-Model ML}} Sederhana (\textit{Logistic Regression}) & Stabil untuk \textit{dataset} kecil, cepat dan ringan, hasil mudah diinterpretasikan, selaras dengan SENSOR DOC. & Kurang mampu menangkap pola \textit{non-linear} pada data sensor. \\
\hline
3 & \textbf{\textit{Multi-Model ML}} Klasik (LR, RF, KNN, DT) & Lebih fleksibel, dapat mengatasi pola \textit{non-linear}, mendukung pemilihan model terbaik melalui evaluasi, cocok untuk \textit{dataset} kecil seperti sensor MQ. & Membutuhkan proses \textit{training}, \textit{tuning}, dan pemetaan performa antar model. \\
\hline
4 & \textbf{\textit{Deep Learning}} (ANN / 1D-CNN) & Performa tinggi pada \textit{dataset} besar, mampu belajar pola kompleks. & Tidak cocok untuk \textit{dataset} kecil, rawan \textit{overfitting}, komputasi besar, tidak selaras dengan target \textit{low-cost Smart Canteen}. \\
\hline
\end{longtable}
