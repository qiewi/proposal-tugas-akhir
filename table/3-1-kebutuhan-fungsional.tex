\begin{longtable}{|p{1.5cm}|p{4cm}|p{8cm}|}
\caption{Kebutuhan Fungsional Sistem}\label{tbl:kebutuhan-fungsional} \\
\hline
\textbf{ID} & \textbf{Nama Fitur} & \textbf{Deskripsi} \\
\hline
\endfirsthead

\caption{Kebutuhan Fungsional Sistem (lanjutan)} \\
\hline
\textbf{ID} & \textbf{Nama Fitur} & \textbf{Deskripsi} \\
\hline
\endhead

\hline
\multicolumn{3}{r}{\textit{Bersambung ke halaman berikutnya}} \\
\endfoot

\hline
\endlastfoot

FR01 & Pemrosesan Data Sensor & Sistem harus mampu membaca dan mengelola input dari sensor MQ (MQ8, MQ135, MQ4, MQ9, MQ2, MQ3) berupa nilai analog dan digital untuk mendeteksi pola gas volatil yang berkaitan dengan \textit{spoilage}. \\
\hline
FR02 & Klasifikasi Kesegaran Makanan & Sistem harus menerapkan model \textit{machine learning} untuk mengklasifikasikan kondisi makanan ke dalam dua kelas, \textit{fresh} atau \textit{spoiled} secara otomatis berdasarkan data sensor. \\
\hline
FR03 & Integrasi Hasil Klasifikasi & Sistem harus mampu menghasilkan rekomendasi keputusan \textit{storing}, \textit{donation}, \textit{disposal} berdasarkan hasil prediksi kesegaran makanan. \\
\hline
FR04 & Visualisasi Hasil & Sistem harus mampu menampilkan status kesegaran makanan dalam bentuk dashboard sederhana (misalnya indikator warna: hijau = \textit{fresh}, merah = \textit{spoiled}). \\
\hline
\end{longtable}
