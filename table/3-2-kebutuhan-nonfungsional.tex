\begin{longtable}{|p{1.5cm}|p{4cm}|p{8cm}|}
\caption{Kebutuhan Nonfungsional Sistem}\label{tbl:kebutuhan-nonfungsional} \\
\hline
\textbf{ID} & \textbf{Aspek Kualitas} & \textbf{Deskripsi} \\
\hline
\endfirsthead

\caption{Kebutuhan Nonfungsional Sistem (lanjutan)} \\
\hline
\textbf{ID} & \textbf{Aspek Kualitas} & \textbf{Deskripsi} \\
\hline
\endhead

\hline
\multicolumn{3}{r}{\textit{Bersambung ke halaman berikutnya}} \\
\endfoot

\hline
\endlastfoot

NFR01 & Keandalan Prediksi & Model klasifikasi harus memiliki akurasi yang memadai (> 85\%) dengan fokus pada penurunan \textit{false negative} (makanan rusak tidak terdeteksi). \\
\hline
NFR02 & Waktu Respons & Sistem harus mampu menghasilkan prediksi dalam waktu < 2 detik per sampel untuk penggunaan operasional \textit{real-time} pada kantin. \\
\hline
NFR03 & Skalabilitas & Sistem mampu menangani input dari beberapa sensor sekaligus dan dapat diperluas untuk jenis makanan baru tanpa perubahan besar pada arsitektur. \\
\hline
NFR04 & Kemudahan Implementasi & Sistem harus berbasis \textit{low-cost hardware} (Arduino + MQ sensors) sehingga dapat diimplementasikan pada kantin sekolah/kampus tanpa biaya tinggi. \\
\hline
\end{longtable}
